%%%%%%%%%%%%%%%%%%%%%%%%%%%%%%%%%%%
\chapter*{\large\g ΙΛΙΑΔΟΣ Α}
%\markboth{ΙΛΙΑΔΟΣ Α}{}
\markright{1. \hspace{1em} ΙΛΙΑΔΟΣ Α}{}
\addcontentsline{toc}{chapter}{Text of ΙΛΙΑΔΟΣ Α}%

%%%%   checked against West's edition of Iliad
\begin{Spacing}{1.5}
\begin{verse}  % book one 
{\large\g μῆνιν ἄειδε, θεά, Πηληϊάδεω Ἀχιλῆος,  } \\
{\large\g οὐλομένην, ἣ μυρί᾽ Ἀχαιοῖς ἄλγε᾽ ἔθηκεν,  } \\
{\large\g πολλὰς δ᾽ ἰφθίμους ψυχὰς Ἄϊδι προΐαψεν   } \\
{\large\g  ἡρώων, αὐτοὺς δὲ ἑλώρια τεῦχε κύνεσσιν  } \\
{\large\g οἰωνοῖσί τε πᾶσι, Διὸς δ᾽ ἐτελείετο βουλή,   } \\
{\large\g ἐξ οὗ δὴ τὰ πρῶτα διαστήτην ἐρίσαντε   } \\
{\large\g Ἀτρείδης τε ἄναξ ἀνδρῶν καὶ δῖος Ἀχιλλεύς.  } \\
{\large\g τίς τάρ σφωε θεῶν ἔριδι ξυνέηκε μάχεσθαι;  } \\
{\large\g Λητοῦς καὶ Διὸς υἱός: ὃ γὰρ βασιλῆϊ χολωθεὶς  } \\
{\large\g νοῦσον ἀνὰ στρατὸν ὦρσε κακήν, ὀλέκοντο δὲ λαοί,  } \\
{\large\g οὕνεκα τὸν Χρύσην ἠτίμασεν ἀρητῆρα  } \\
{\large\g Ἀτρείδης. ὃ γὰρ ἦλθε θοὰς ἐπὶ νῆας Ἀχαιῶν  } \\
{\large\g λυσόμενός τε θύγατρα φέρων τ᾽ ἀπερείσι᾽ ἄποινα,  } \\
{\large\g στέμματ᾽ ἔχων ἐν χερσὶν ἑκηβόλου Ἀπόλλωνος  } \\
{\large\g χρυσέωι ἀνὰ σκήπτρωι, καὶ λίσσετο πάντας Ἀχαιούς,  } \\
{\large\g Ἀτρεΐδα δὲ μάλιστα δύω, κοσμήτορε λαῶν:  } \\
{\large\g  Ἀτρεΐδαι τε καὶ ἄλλοι ἐϋκνήμιδες Ἀχαιοί,   } \\
{\large\g ὑμῖν μὲν θεοὶ δοῖεν Ὀλύμπια δώματ᾽ ἔχοντες   } \\
{\large\g  ἐκπέρσαι Πριάμοιο πόλιν, εὖ δ᾽ οἴκαδ᾽ ἱκέσθαι:  } \\
{\large\g παῖδα δ᾽ ἐμοὶ λύσαιτε φίλην, τὰ δ᾽ ἄποινα δέχεσθαι,   } \\
{\large\g ἁζόμενοι Διὸς υἱὸν ἑκηβόλον Ἀπόλλωνα.   } \\
{\large\g  ἔνθ᾽ ἄλλοι μὲν πάντες ἐπευφήμησαν Ἀχαιοὶ,  } \\
{\large\g  αἰδεῖσθαί θ᾽ ἱερῆα καὶ ἀγλαὰ δέχθαι ἄποινα:  } \\
{\large\g  ἀλλ᾽ οὐκ Ἀτρεΐδηι Ἀγαμέμνονι ἥνδανε θυμῶι,  } \\
{\large\g  ἀλλὰ κακῶς ἀφίει, κρατερὸν δ᾽ ἐπὶ μῦθον ἔτελλεν:  } \\
{\large\g  μή σε, γέρον, κοίληισιν ἐγὼ παρὰ νηυσὶ κιχείω  } \\
{\large\g  ἢ νῦν δηθύνοντ᾽ ἢ ὕστερον αὖτις ἰόντα,  } \\
{\large\g  μή νύ τοι οὐ χραίσμηι σκῆπτρον καὶ στέμμα θεοῖο.  } \\
{\large\g  τὴν δ᾽ ἐγὼ οὐ λύσω: πρίν μιν καὶ γῆρας ἔπεισιν  } \\
{\large\g  ἡμετέρωι ἐνὶ οἴκωι, ἐν Ἄργεϊ, τηλόθι πάτρης,  } \\
{\large\g  ἱστὸν ἐποιχομένην καὶ ἐμὸν λέχος ἀντιόωσαν.  } \\
{\large\g  ἀλλ᾽ ἴθι, μή μ᾽ ἐρέθιζε, σαώτερος ὥς κε νέηαι.  } \\
{\large\g  ὣς ἔφατ᾽. ἔδδεισεν δ᾽ ὃ γέρων καὶ ἐπείθετο μύθωι.  } \\
{\large\g  βῆ δ᾽ ἀκέων παρὰ θῖνα πολυφλοίσβοιο θαλάσσης.  } \\
{\large\g  πολλὰ δ᾽ ἔπειτ᾽ ἀπάνευθε κιὼν ἠρᾶθ᾽ ὃ γεραιὸς  } \\
{\large\g  Ἀπόλλωνι ἄνακτι, τὸν ἠΰκομος τέκε Λητώ:  } \\
{\large\g  κλῦθί μοι, ἀργυρότοξ᾽, ὃς Χρύσην ἀμφιβέβηκας  } \\
{\large\g  Κίλλάν τε ζαθέην, Τενέδοιό τε ἶφι ἀνάσσεις,  } \\
{\large\g  Σμινθεῦ: εἴ ποτέ τοι χαρίεντ᾽ ἐπὶ νηὸν ἔρεψα,  } \\
{\large\g  ἢ εἰ δή ποτέ τοι κατὰ πίονα μηρί᾽ ἔκηα  } \\
{\large\g  ταύρων ἠδ᾽ αἰγῶν, τόδε μοι κρήηνον ἐέλδωρ:  } \\
{\large\g  τίσειαν Δαναοὶ ἐμὰ δάκρυα σοῖσι βέλεσσιν.  } \\
{\large\g  ὣς ἔφατ᾽ εὐχόμενος, τοῦ δ᾽ ἔκλυε Φοῖβος Ἀπόλλων,  } \\
{\large\g  βῆ δὲ κατ᾽ Οὐλύμποιο καρήνων χωόμενος κῆρ,  } \\
{\large\g  τόξ᾽ ὤμοισιν ἔχων ἀμφηρεφέα τε φαρέτρην:  } \\
{\large\g  ἔκλαγξαν δ᾽ ἄρ᾽ ὀϊστοὶ ἐπ᾽ ὤμων χωομένοιο,  } \\
{\large\g  αὐτοῦ κινηθέντος: ὃ δ᾽ ἤϊε νυκτὶ ἐοικώς.  } \\
{\large\g  ἕζετ᾽ ἔπειτ᾽ ἀπάνευθε νεῶν, μετὰ δ᾽ ἰὸν ἕηκε,  } \\
{\large\g  δεινὴ δὲ κλαγγὴ γένετ᾽ ἀργυρέοιο βιοῖο.  } \\
{\large\g  οὐρῆας μὲν πρῶτον ἐπωίχετο καὶ κύνας ἀργούς,  } \\
{\large\g  αὐτὰρ ἔπειτ᾽ αὐτοῖσι βέλος ἐχεπευκὲς ἐφιεὶς  } \\
{\large\g  βάλλ᾽: αἰεὶ δὲ πυραὶ νεκύων καίοντο θαμειαί.  } \\
{\large\g  ἐννῆμαρ μὲν ἀνὰ στρατὸν ωἴχετο κῆλα θεοῖο,  } \\
{\large\g  τῆι δεκάτηι δ᾽ ἀγορὴν δὲ καλέσσατο λαὸν Ἀχιλλεύς:  } \\
{\large\g  τῶι γὰρ ἐπὶ φρεσὶ θῆκε θεὰ λευκώλενος Ἥρη:  } \\
{\large\g  κήδετο γὰρ Δαναῶν, ὅτι ῥα θνήσκοντας ὁρᾶτο.  } \\
{\large\g  οἳ δ᾽ ἐπεὶ οὖν ἤγερθεν ὁμηγερέες τ᾽ ἐγένοντο,  } \\
{\large\g  τοῖσι δ᾽ ἀνιστάμενος μετέφη πόδας ὠκὺς Ἀχιλλεύς:  } \\
{\large\g  Ἀτρεΐδη, νῦν ἄμμε παλιν πλαγχθέντας ὀΐω  } \\
{\large\g  ἂψ ἀπονοστήσειν, εἴ κεν θάνατόν γε φύγοιμεν,  } \\
{\large\g  εἰ δὴ ὁμοῦ πόλεμός τε δαμᾶι καὶ λοιμὸς Ἀχαιούς.  } \\
{\large\g  ἀλλ᾽ ἄγε δή τινα μάντιν ἐρείομεν ἠ᾽ ἱερῆα,  } \\
{\large\g  ἢ καὶ ὀνειροπόλον, καὶ γάρ τ᾽ ὄναρ ἐκ Διός ἐστιν,  } \\
{\large\g  ὅς κ᾽ εἴποι ὅ τι τόσσον ἐχώσατο Φοῖβος Ἀπόλλων,  } \\
{\large\g  εἴ ταρ ὅ γ᾽ εὐχωλῆς ἐπιμέμφεται ἠδ᾽ ἑκατόμβης,  } \\
{\large\g  αἴ κέν πως ἀρνῶν κνίσης αἰγῶν τε τελείων  } \\
{\large\g  βούλητ᾽ ἀντιάσας ἡμῖν ἀπὸ λοιγὸν ἀμῦναι.  } \\
{\large\g  ἤτοι ὅ γ᾽ ὣς εἰπὼν κατ᾽ ἄρ᾽ ἕζετο: τοῖσι δ᾽ ἀνέστη  } \\
{\large\g  Κάλχας Θεστορίδης, οἰωνοπόλων ὄχ᾽ ἄριστος,  } \\
{\large\g  ὃς εἴδη τά τ᾽ ἐόντα τά τ᾽ ἐσσόμενα πρό τ᾽ ἐόντα,  } \\
{\large\g  καὶ νήεσσ᾽ ἡγήσατ᾽ Ἀχαιῶν Ἴλιον εἴσω  } \\
{\large\g  ἣν διὰ μαντοσύνην, τήν οἱ πόρε Φοῖβος Ἀπόλλων.  } \\
{\large\g  ὅ σφιν ἔϋ φρονέων ἀγορήσατο καὶ μετέειπεν:  } \\
{\large\g  ὦ Ἀχιλεῦ, κέλεαί με, διίφιλε, μυθήσασθαι  } \\
{\large\g  μῆνιν Ἀπόλλωνος ἑκατηβελέταο ἄνακτος.  } \\
{\large\g  τοὶ γὰρ ἐγὼν ἐρέω, σὺ δὲ σύνθεο καὶ μοι ὄμοσσον  } \\
{\large\g  ἦ μέν μοι πρόφρων ἔπεσιν καὶ χερσὶν ἀρήξειν:  } \\
{\large\g  ἦ γὰρ ὀΐομαι ἄνδρα χολωσέμεν, ὃς μέγα πάντων  } \\
{\large\g  Ἀργείων κρατέει καί οἱ πείθονται Ἀχαιοί.  } \\
{\large\g  κρέσσων γὰρ βασιλεύς, ὅτε χώσεται ἀνδρὶ χέρηϊ:  } \\
{\large\g  εἴ περ γάρ τε χόλον γε καὶ αὐτῆμαρ καταπέψηι,  } \\
{\large\g  ἀλλά τε καὶ μετόπισθεν ἔχει κότον, ὄφρα τελέσσηι,  } \\
{\large\g  ἐν στήθεσσιν ἑοῖσι. σὺ δὲ φράσαι εἴ με σαώσεις.  } \\
{\large\g  τὸν δ᾽ ἀπαμειβόμενος προσέφη πόδας ὠκὺς Ἀχιλλεύς:  } \\
{\large\g  θαρσήσας μάλα εἰπὲ θεοπρόπιον ὅ τι οἶσθα:  } \\
{\large\g  οὐ μὰ γὰρ Ἀπόλλωνα διίφιλον, ὧι τε σύ, Κάλχαν,  } \\
{\large\g  εὐχόμενος Δαναοῖσι θεοπροπίας ἀναφαίνεις,  } \\
{\large\g  οὔ τις ἐμέο ζῶντος καὶ ἐπὶ χθονὶ δερκομένοιο  } \\
{\large\g  σοὶ κοίληις παρὰ νηυσί βαρείας χεῖρας ἐποίσει  } \\
{\large\g  συμπάντων Δαναῶν, οὐδ᾽ ἢν Ἀγαμέμνονα εἴπηις,  } \\
{\large\g  ὃς νῦν πολλὸν ἄριστος ἐνὶ στρατῶι εὔχεται εἶναι.  } \\
{\large\g  καὶ τότε δὴ θάρσησε καὶ ηὔδα μάντις ἀμύμων:  } \\
{\large\g  οὔτ᾽ ἄρ᾽ ὅ γ᾽ εὐχωλῆς ἐπιμέμφεται οὐδ᾽ ἑκατόμβης,  } \\
{\large\g  ἀλλ᾽ ἕνεκ᾽ ἀρητῆρος, ὃν ἠτίμησ᾽ Ἀγαμέμνων  } \\
{\large\g  οὐδ᾽ ἀπέλυσε θύγατρα καὶ οὐκ ἀπεδέξατ᾽ ἄποινα,  } \\
{\large\g  τοὔνεκ᾽ ἄρ᾽ ἄλγε᾽ ἔδωκεν Ἑκηβόλος, ἠδ᾽ ἔτι δώσει:  } \\
{\large\g  οὐδ᾽ ὅ γε πρὶν λοιμοῖο βαρείας χεῖρας ἀφέξει,  } \\
{\large\g  πρίν γ᾽ ἀπὸ πατρὶ φίλωι δόμεναι ἑλικώπιδα κούρην  } \\
{\large\g  ἀπριάτην ἀνάποινον, ἄγειν θ᾽ ἱερὴν ἑκατόμβην  } \\
{\large\g  ἐς Χρύσην: τότε κέν μιν ἱλασσάμενοι πεπίθοιμεν.  } \\
{\large\g  ἤτοι ὅ γ᾽ ὣς εἰπὼν κατ᾽ ἄρ᾽ ἕζετο: τοῖσι δ᾽ ἀνέστη  } \\
{\large\g  ἥρως Ἀτρεΐδης, εὐρὺ κρείων Ἀγαμέμνων,  } \\
{\large\g  ἀχνύμενος: μένεος δὲ μέγα φρένες ἀμφι μέλαιναι  } \\
{\large\g  πίμπλαντ᾽, ὄσσε δέ οἱ πυρὶ λαμπετόωντι ἐΐκτην.  } \\
{\large\g  Κάλχαντα πρώτιστα κάκ᾽ ὀσσόμενος προσέειπεν:  } \\
{\large\g  μάντι κακῶν, οὔ πώ ποτέ μοι τὸ κρήγυον εἶπες:  } \\
{\large\g  αἰεί τοι τὰ κάκ᾽ ἐστὶ φίλα φρεσὶ μαντεύεσθαι,  } \\
{\large\g  ἐσθλὸν δ᾽ οὔτέ τί πω εἶπες ἔπος οὔτ᾽ ἐτέλεσσας.  } \\
{\large\g  καὶ νῦν ἐν Δαναοῖσι θεοπροπέων ἀγορεύεις,  } \\
{\large\g  ὡς δὴ τοῦδ᾽ ἕνεκά σφιν Ἑκηβόλος ἄλγεα τεύχει,  } \\
{\large\g  οὕνεκ᾽ ἐγὼ κούρης Χρυσηΐδος ἀγλά᾽ ἄποινα  } \\
{\large\g  οὐκ ἔθελον δέξασθαι, ἐπεὶ πολὺ βούλομαι αὐτήν  } \\
{\large\g  οἴκοι ἔχειν: καὶ γάρ ῥα Κλυταιμνήστρης προβέβουλα  } \\
{\large\g  κουριδίης ἀλόχου, ἐπεὶ οὔ ἑθέν ἐστι χερείων,  } \\
{\large\g  οὐ δέμας οὐδὲ φυήν, οὔτ᾽ ἂρ φρένας οὔτέ τι ἔργα.  } \\
{\large\g ἀλλὰ καὶ ὧς ἐθέλω δόμεναι πάλιν, εἰ τό γ᾽ ἄμεινον:   } \\
{\large\g  βούλομ᾽ ἐγὼ λαὸν σόον ἔμμεναι ἠ᾽ ἀπολέσθαι.  } \\
{\large\g  αὐτὰρ ἐμοὶ γέρας αὐτίχ᾽ ἑτοιμάσατ᾽, ὄφρα μὴ οἶος  } \\
{\large\g  Ἀργείων ἀγέραστος ἔω, ἐπεὶ οὐδὲ ἔοικεν:  } \\
{\large\g  λεύσσετε γὰρ τό γε πάντες, ὅ μοι γέρας ἔρχεται ἄλληι.  } \\
{\large\g  τὸν δ᾽ ἠμείβετ᾽ ἔπειτα ποδάρκης δῖος Ἀχιλλεύς:  } \\
{\large\g  Ἀτρεΐδη κύδιστε, φιλοκτεανώτατε πάντων,  } \\
{\large\g  πῶς τάρ τοι δώσουσι γέρας μεγάθυμοι Ἀχαιοί;  } \\
{\large\g  οὐδέ τί που ἴδμεν ξυνήϊα κείμενα πολλά:  } \\
{\large\g  ἀλλὰ τὰ μὲν πολίων ἐξεπράθομεν, τὰ δέδασται,  } \\
{\large\g  λαοὺς δ᾽ οὐκ ἐπέοικε παλίλλογα ταῦτ᾽ ἐπαγείρειν.  } \\
{\large\g  ἀλλὰ σὺ μὲν νῦν τήνδε θεῶι πρόες: αὐτὰρ Ἀχαιοί  } \\
{\large\g  τριπλῆι τετραπλῆι τ᾽ ἀποτείσομεν, αἴ κέ ποθι Ζεύς  } \\
{\large\g  δῶσι πόλιν Τροίην εὐτείχεον ἐξαλαπάξαι.  } \\
{\large\g  τὸν δ᾽ ἀπαμειβόμενος προσέφη κρείων Ἀγαμέμνων:  } \\
{\large\g  μὴ δὴ οὕτως, ἀγαθός περ ἐὼν, θεοείκελ᾽ Ἀχιλλεῦ,  } \\
{\large\g  κλέπτε νόωι, ἐπεὶ οὐ παρελεύσεαι οὐδέ με πείσεις.  } \\
{\large\g  ἦ ἐθέλεις, ὄφρ᾽ αὐτὸς ἔχηις γέρας, αὐτὰρ ἔμ᾽ αὔτως  } \\
{\large\g  ἧσθαι δευόμενον, κέλεαι δέ με τήνδ᾽ ἀποδοῦναι;  } \\
{\large\g  ἀλλ᾽ εἰ μὲν δώσουσι γέρας μεγάθυμοι Ἀχαιοί,  } \\
{\large\g  ἄρσαντες κατὰ θυμὸν, ὅπως ἀντάξιον ἔσται:  } \\
{\large\g  εἰ δέ κε μὴ δώωσιν, ἐγὼ δέ κεν αὐτὸς ἕλωμαι  } \\
{\large\g  ἢ τεὸν ἠ᾽ Αἴαντος ἰὼν γέρας, ἢ Ὀδυσῆος  } \\
{\large\g  ἄξω ἑλών: ὃ δέ κεν κεχολώσεται ὅν κεν ἵκωμαι.  } \\
{\large\g  ἀλλ᾽ ἤτοι μὲν ταῦτα μεταφρασόμεσθα καὶ αὖτις,  } \\
{\large\g  νῦν δ᾽ ἄγε νῆα μέλαιναν ἐρύσσομεν εἰς ἅλα δῖαν,  } \\
{\large\g  ἐν δ᾽ ἐρέτας ἐπιτηδὲς ἀγείρομεν, ἐς δ᾽ ἑκατόμβην  } \\
{\large\g  θείομεν, ἂν δ᾽ αὐτὴν Χρυσηΐδα καλλιπάρηιον  } \\
{\large\g  βήσομεν: εἷς δέ τις ἀρχὸς ἀνὴρ βουληφόρος ἔστω,  } \\
{\large\g  ἠ᾽ Αἴας ἠ᾽ Ἰδομενεὺς ἢ δῖος Ὀδυσσεύς,  } \\
{\large\g  ἠὲ σύ, Πηλεΐδη, πάντων ἐκπαγλότατ᾽ ἀνδρῶν,  } \\
{\large\g  ὄφρ᾽ ἥμιν ἑκάεργον ἱλάσσεαι ἱερὰ ῥέξας.  } \\
{\large\g  τὸν δ᾽ ἄρ᾽ ὑπόδρα ἰδὼν προσέφη πόδας ὠκὺς Ἀχιλλεύς:  } \\
{\large\g  ωἴ μοι, ἀναιδείην ἐπιειμένε, κερδαλεόφρον,  } \\
{\large\g  πῶς τίς τοι πρόφρων ἔπεσιν πείθηται Ἀχαιῶν,  } \\
{\large\g  ἢ ὁδὸν ἐλθέμεναι ἠ᾽ ἀνδράσιν ἶφι μάχεσθαι;  } \\
{\large\g  οὐ γὰρ ἐγὼ Τρώων ἕνεκ᾽ ἤλυθον αἰχμητάων  } \\
{\large\g  δεῦρο μαχησόμενος, ἐπεὶ οὔ τί μοι αἴτιοί εἰσιν:  } \\
{\large\g  οὐ γὰρ πώ ποτ᾽ ἐμὰς βοῦς ἤλασαν οὐδὲ μὲν ἵππους,  } \\
{\large\g  οὐδέ ποτ᾽ ἐν Φθίηι ἐριβώλακι βωτιανείρηι  } \\
{\large\g  καρπὸν ἐδηλήσαντ᾽, ἐπεὶ ἦ μάλα πολλὰ μεταξύ,  } \\
{\large\g  οὔρεά τε σκιόεντα θάλασσά τε ἠχήεσσα.  } \\
{\large\g  ἀλλὰ σοί, ὦ μέγ᾽ ἀναιδές, ἅμ᾽ ἑσπόμεθ᾽, ὄφρα σὺ χαίρηις,  } \\
{\large\g  τιμὴν ἀρνύμενοι Μενελάωι σοί τε, κυνῶπα,  } \\
{\large\g  πρὸς Τρώων: τῶν οὔ τι μετατρέπε᾽ οὐδ᾽ ἀλεγίζεις.  } \\
{\large\g  καὶ δή μοι γέρας αὐτὸς ἀφαιρήσεσθαι ἀπειλεῖς,  } \\
{\large\g  ὧι ἔπι πόλλ᾽ ἐμόγησα, δόσαν δέ μοι υἷες Ἀχαιῶν.  } \\
{\large\g οὐ μὲν σοί ποτε ἶσον ἔχω γέρας, ὁππότ᾽ Ἀχαιοί   } \\
{\large\g  Τρώων ἐκπέρσωσ᾽ εὖ ναιόμενον πτολίεθρον,  } \\
{\large\g  ἀλλὰ τὸ μὲν πλεῖον πολυάϊκος πολέμοιο  } \\
{\large\g  χεῖρες ἐμαὶ διέπουσ᾽, ἀτὰρ ἤν ποτε δασμὸς ἵκηται,  } \\
{\large\g  σοὶ τὸ γέρας πολὺ μέζον, ἐγὼ δ᾽ ὀλίγον τε φίλον τε  } \\
{\large\g  ἔρχομ᾽ ἔχων ἐπὶ νῆας, ἐπεί κε κάμω πολεμίζων.  } \\
{\large\g  νῦν δ᾽ εἶμι Φθίηνδ᾽, ἐπεὶ ἦ πολὺ φέρτερόν ἐστιν  } \\
{\large\g  οἴκαδ᾽ ἴμεν σὺν νηυσὶ κορωνίσιν, οὐδέ σ᾽ ὀΐω  } \\
{\large\g  ἐνθάδ᾽ ἄτιμος ἐὼν ἄφενος καὶ πλοῦτον ἀφύξειν.  } \\
{\large\g τὸν δ᾽ ἠμείβετ᾽ ἔπειτα ἄναξ ἀνδρῶν Ἀγαμέμνων:  } \\
{\large\g φεῦγε μάλ᾽, εἴ τοι θυμὸς ἐπέσσυται. οὐδέ σ᾽ ἔγω γε  } \\
{\large\g λίσσομαι εἵνεκ᾽ ἐμεῖο μένειν: πάρ᾽ ἐμοί γε καὶ ἄλλοι  } \\
{\large\g οἵ κέ με τιμήσουσι, μάλιστα δὲ μητίετα Ζεύς.  } \\
{\large\g ἔχθιστος δέ μοί ἐσσι διοτρεφέων βασιλήων:  } \\
{\large\g αἰεὶ γάρ τοι ἔρις τε φίλη πόλεμοί τε μάχαι τε.  } \\
{\large\g εἰ μάλα καρτερός ἐσσι, θεός που σοὶ τό γ᾽ ἔδωκεν.  } \\
{\large\g οἴκαδ᾽ ἰὼν σὺν νηυσί τε σῆις καὶ σοῖς ἑτάροισι  } \\
{\large\g Μυρμιδόνεσσιν ἄνασσε: σέθεν δ᾽ ἐγὼ οὐκ ἀλεγίζω  } \\
{\large\g οὐδ᾽ ὄθομαι κοτέοντος. ἀπειλήσω δέ τοι ὧδε:  } \\
{\large\g ὡς ἔμ᾽ ἀφαιρεῖται Χρυσηΐδα Φοῖβος Ἀπόλλων,  } \\
{\large\g τὴν μὲν ἐγὼ σὺν νηΐ τ᾽ ἐμῆι καὶ ἐμοῖς ἑτάροισι  } \\
{\large\g πέμψω: ἐγὼ δέ κ᾽ ἄγω Βρισηΐδα καλλιπάρηον  } \\
{\large\g αὐτὸς ἰὼν κλισίηνδε, τεὸν γέρας, ὄφρ᾽ εὖ εἰδῆις  } \\
{\large\g ὅσσον φέρτερός εἰμι σέθεν, στυγέηι δὲ καὶ ἄλλος  } \\
{\large\g ἶσον ἐμοὶ φάσθαι καὶ ὁμοιωθήμεναι ἄντην.  } \\
{\large\g ὣς φάτο: Πηλεΐωνι δ᾽ ἄχος γένετ᾽, ἐν δέ οἱ ἦτορ  } \\
{\large\g στήθεσσιν λασίοισι διάνδιχα μερμήριξεν,  } \\
{\large\g ἠ᾽ ὅ γε φάσγανον ὀξὺ ἐρυσσάμενος παρὰ μηροῦ  } \\
{\large\g τοὺς μὲν ἀναστήσειεν, ὃ δ᾽ Ἀτρεΐδην ἐναρίζοι,  } \\
{\large\g ἦε χόλον παύσειεν ἐρητύσειέ τε θυμόν.  } \\
{\large\g ἕως ὃ ταῦθ᾽ ὥρμαινε κατὰ φρένα καὶ κατὰ θυμόν,  } \\
{\large\g εἵλκετο δ᾽ ἐκ κολεοῖο μέγα ξίφος, ἦλθε δ᾽ Ἀθήνη  } \\
{\large\g οὐρανόθεν: πρὸ γὰρ ἧκε θεὰ λευκώλενος Ἥρη,  } \\
{\large\g ἄμφω ὁμῶς θυμῶι φιλέουσά τε κηδομένη τε.  } \\
{\large\g στῆ δ᾽ ὄπιθεν, ξανθῆς δὲ κόμης ἕλε Πηλεΐωνα,  } \\
{\large\g οἴωι φαινομένη, τῶν δ᾽ ἄλλων οὔ τις ὁρᾶτο.  } \\
{\large\g θάμβησεν δ᾽ Ἀχιλεύς, μετὰ δ᾽ ἐτράπετ᾽: αὐτίκα δ᾽ ἔγνω  } \\
{\large\g Παλλάδ᾽ Ἀθηναίην: δεινὼ δέ οἱ ὄσσε φάανθεν.  } \\
{\large\g καί μιν φωνήσας ἔπεα πτερόεντα προσηύδα:  } \\
{\large\g τίπτ᾽ αὖτ᾽, αἰγιόχοιο Διὸς τέκος, εἰλήλουθας;  } \\
{\large\g ἦ ἵνα ὕβριν ἴδη᾽  Ἀγαμέμνονος Ἀτρεΐδαο;  } \\
{\large\g ἀλλ᾽ ἔκ τοι ἐρέω, τὸ δὲ καὶ τελέεσθαι ὀΐω:  } \\
{\large\g ἧις ὑπεροπλίηισι τάχ᾽ ἄν ποτε θυμὸν ὀλέσσηι.  } \\
{\large\g τὸν δ᾽ αὖτε προσέειπε θεὰ γλαυκῶπις Ἀθήνη:  } \\
{\large\g ἦλθον ἐγὼ παύσουσα τεὸν μένος, αἴ κε πίθηαι,  } \\
{\large\g οὐρανόθεν: πρὸ δέ μ᾽ ἧκε θεὰ λευκώλενος Ἥρη,  } \\
{\large\g ἄμφω ὁμῶς θυμῶι φιλέουσά τε κηδομένη τε.  } \\
{\large\g ἀλλ᾽ ἄγε λῆγ᾽ ἔριδος, μηδὲ ξίφος ἕλκεο χειρί:  } \\
{\large\g ἀλλ᾽ ἤτοι ἔπεσιν μὲν ὀνείδισον, ὡς ἔσεταί περ.  } \\
{\large\g ὧδε γὰρ ἐξερέω, τὸ δὲ καὶ τετελεσμένον ἔσται:  } \\
{\large\g καί ποτέ τοι τρὶς τόσσα παρέσσεται ἀγλαὰ δῶρα  } \\
{\large\g ὕβριος εἵνεκα τῆσδε: σὺ δ᾽ ἴσχεο, πείθεο δ᾽ ἡμῖν.  } \\
{\large\g τὴν δ᾽ ἀπαμειβόμενος προσέφη πόδας ὠκὺς Ἀχιλλεύς:  } \\
{\large\g χρὴ μὲν σφωΐτερόν γε, θεὰ, ἔπος εἰρύσσασθαι,  } \\
{\large\g καὶ μάλα περ θυμῶι κεχολωμένον: ὧς γὰρ ἄμεινον.  } \\
{\large\g ὅς κε θεοῖς ἐπιπείθηται, μάλα τ᾽ ἔκλυον αὐτοῦ.  } \\
{\large\g ἦ, καὶ ἐπ᾽ ἀργυρέηι κώπηι σχέθε χεῖρα βαρεῖαν,  } \\
{\large\g ἂψ δ᾽ ἐς κουλεὸν ὦσε μέγα ξίφος, οὐδ᾽ ἀπίθησε  } \\
{\large\g μύθωι Ἀθηναίης: ἣ δ᾽ Οὔλυμπόνδε βεβήκει  } \\
{\large\g δώματ᾽ ἐς αἰγιόχοιο Διὸς μετὰ δαίμονας ἄλλους.  } \\
{\large\g Πηλεΐδης δ᾽ ἐξαῦτις ἀταρτηροῖς ἐπέεσσιν  } \\
{\large\g Ἀτρεΐδην προσέειπε, καὶ οὔ πω λῆγε χόλοιο:  } \\
{\large\g οἰνοβαρές, κυνὸς ὄμματ᾽ ἔχων, κραδίην δ᾽ ἐλάφοιο,  } \\
{\large\g οὔτέ ποτ᾽ ἐς πόλεμον ἅμα λαῶι θωρηχθῆναι  } \\
{\large\g οὔτε λόχονδ᾽ ἰέναι σὺν ἀριστήεσσιν Ἀχαιῶν  } \\
{\large\g τέτληκας θυμῶι: τὸ δέ τοι κὴρ εἴδεται εἶναι.  } \\
{\large\g ἦ πολὺ λώϊόν ἐστι κατὰ στρατὸν εὐρὺν Ἀχαιῶν  } \\
{\large\g δῶρ᾽ ἀποαιρεῖσθαι, ὅς τις σέθεν ἀντίον εἴπηι.  } \\
{\large\g δημοβόρος βασιλεύς, ἐπεὶ οὐτιδανοῖσιν ἀνάσσεις:  } \\
{\large\g ἦ γὰρ ἄν, Ἀτρεΐδη νῦν ὕστατα λωβήσαιο.  } \\
{\large\g ἀλλ᾽ ἔκ τοι ἐρέω, καὶ ἐπὶ μέγαν ὅρκον ὀμοῦμαι:  } \\
{\large\g ναὶ μὰ τόδε σκῆπτρον: τὸ μὲν οὔ ποτε φύλλα καὶ ὄζους  } \\
{\large\g φύσει, ἐπεὶ δὴ πρῶτα τομὴν ἐν ὄρεσσι λέλοιπεν,  } \\
{\large\g οὐδ᾽ ἀναθηλήσει: περὶ γάρ ῥά ἑ χαλκὸς ἔλεψεν  } \\
{\large\g φύλλά τε καὶ φλοιόν: νῦν αὖτέ μιν υἷες Ἀχαιῶν  } \\
{\large\g ἐν παλάμαις φορέουσι δικασπόλοι, οἵ τε θέμιστας  } \\
{\large\g πρὸς Διὸς εἰρύαται: ὃ δέ τοι μέγας ἔσσεται ὅρκος:  } \\
{\large\g ἦ ποτ᾽ Ἀχιλλῆος ποθὴ ἵξεται υἷας Ἀχαιῶν  } \\
{\large\g σύμπαντας: τότε δ᾽ οὔ τι δυνήσεαι ἀχνύμενός περ  } \\
{\large\g χραισμεῖν, εὖτ᾽ ἂν πολλοὶ ὑφ᾽ Ἕκτορος ἀνδροφόνοιο  } \\
{\large\g θνήσκοντες πίπτωσι: σὺ δ᾽ ἔνδοθι θυμὸν ἀμύξεις  } \\
{\large\g χωόμενος, ὅ τ᾽ ἄριστον Ἀχαιῶν οὐδὲν ἔτισας.  } \\
{\large\g ὣς φάτο Πηλεΐδης, ποτὶ δὲ σκῆπτρον βάλε γαίηι   } \\
{\large\g χρυσείοις ἥλοισι πεπαρμένον, ἕζετο δ᾽ αὐτός:  } \\
{\large\g Ἀτρεΐδης δ᾽ ἑτέρωθεν ἐμήνιε. τοῖσι δὲ Νέστωρ  } \\
{\large\g ἡδυεπὴς ἀνόρουσε, λιγὺς Πυλίων ἀγορητής,  } \\
{\large\g τοῦ καὶ ἀπὸ γλώσσης μέλιτος γλυκίων ῥέεν αὐδή.  } \\
{\large\g τῶι δ᾽ ἤδη δύο μὲν γενεαὶ μερόπων ἀνθρώπων  } \\
{\large\g ἐφθίαθ᾽, οἵ οἱ πρόσθεν ἅμα τράφεν ἠδ᾽ ἐγένοντο  } \\
{\large\g ἐν Πύλωι ἠγαθέηι, μετὰ δὲ τριτάτοισιν ἄνασσεν.  } \\
{\large\g  ὅ σφιν ἔϋ φρονέων ἀγορήσατο καὶ μετέειπεν:  } \\
{\large\g ὦ πόποι, ἦ μέγα πένθος Ἀχαιΐδα γαῖαν ἱκάνει.  } \\
{\large\g ἦ κεν γηθήσαι Πρίαμος Πριάμοιό τε παῖδες,  } \\
{\large\g ἄλλοι τε Τρῶες μέγα κεν κεχαροίατο θυμῶι,  } \\
{\large\g εἰ σφῶϊν τάδε πάντα πυθοίατο μαρναμένοιϊν,  } \\
{\large\g οἳ περὶ μὲν βουλὴν Δαναῶν, περὶ δ᾽ ἐστὲ μάχεσθαι.  } \\
{\large\g ἀλλὰ πίθεσθ᾽: ἄμφω δὲ νεωτέρω ἐστὸν ἐμεῖο.  } \\
{\large\g ἤδη γάρ ποτ᾽ ἐγὼ καὶ ἀρείοσιν ἠέ περ ὑμῖν  } \\
{\large\g ἀνδράσιν ὡμίλησα, καὶ οὔ ποτέ μ᾽ οἵ γ᾽ ἀθέριζον.  } \\
{\large\g οὐ γάρ πω τοίους ἴδον ἀνέρας, οὐδὲ ἴδωμαι,  } \\
{\large\g οἷον Πειρίθοόν τε Δρύαντά τε ποιμένα λαῶν  } \\
{\large\g Καινέα τ᾽ Ἐξάδιόν τε καὶ ἀντίθεον Πολύφημον.  } \\
{\large\g Θησέα τ᾽ Αἰγεΐδην, ἐπιείκελον ἀθανάτοισι:  } \\
{\large\g κάρτιστοι δὴ κεῖνοι ἐπιχθονίων τράφον ἀνδρῶν:  } \\
{\large\g κάρτιστοι μὲν ἔσαν καὶ καρτίστοις ἐμάχοντο,  } \\
{\large\g φηρσὶν ὀρεσκώιοισι, καὶ ἐκπάγλως ἀπόλεσσαν.  } \\
{\large\g καὶ μὲν τοῖσιν ἐγὼ μεθομίλεον ἐκ Πύλου ἐλθὼν,  } \\
{\large\g τηλόθεν ἐξ ἀπίης γαίης: καλέσαντο γὰρ αὐτοί.  } \\
{\large\g καὶ μαχόμην κατ᾽ ἔμ᾽ αὐτὸν ἐγώ: κείνοισι δ᾽ ἂν οὔ τις  } \\
{\large\g τῶν οἳ νῦν βροτοί εἰσιν ἐπιχθόνιοι μαχέοιτο.  } \\
{\large\g καὶ μέν μεο βουλέων ξύνιεν πείθοντό τε μύθωι.  } \\
{\large\g ἀλλὰ πίθεσθε καὶ ὔμμες, ἐπεὶ πείθεσθαι ἄμεινον:  } \\
{\large\g μήτε σὺ τόνδ᾽ ἀγαθός περ ἐὼν ἀποαίρεο κούρην,  } \\
{\large\g ἀλλ᾽ ἔα, ὥς οἱ πρῶτα δόσαν γέρας υἷες Ἀχαιῶν:  } \\
{\large\g μήτε σὺ, Πηλεΐδη, ἔθελ᾽ ἐριζέμεναι βασιλῆϊ  } \\
{\large\g ἀντιβίην, ἐπεὶ οὔ ποθ᾽ ὁμοίης ἔμμορε τιμῆς  } \\
{\large\g σκηπτοῦχος βασιλεύς, ὧι τε Ζεὺς κῦδος ἔδωκεν.  } \\
{\large\g εἰ δὲ σὺ καρτερός ἐσσι θεὰ δέ σε γείνατο μήτηρ,  } \\
{\large\g ἀλλ᾽ ὅδε φέρτερός ἐστιν, ἐπεὶ πλεόνεσσιν ἀνάσσει.  } \\
{\large\g Ἀτρεΐδη, σὺ δὲ παῦε τεὸν μένος: αὐτὰρ ἐγώ γε  } \\
{\large\g λίσσομ᾽ Ἀχιλλῆϊ μεθέμεν χόλον, ὃς μέγα πᾶσιν  } \\
{\large\g ἕρκος Ἀχαιοῖσιν πέλεται πολέμοιο κακοῖο.  } \\
{\large\g τὸν δ᾽ ἀπαμειβόμενος προσέφη κρείων Ἀγαμέμνων:  } \\
{\large\g ναὶ δὴ ταῦτά γε πάντα, γέρον, κατὰ μοῖραν ἔειπες:  } \\
{\large\g ἀλλ᾽ ὅδ᾽ ἀνὴρ ἐθέλει περὶ πάντων ἔμμεναι ἄλλων:  } \\
{\large\g πάντων μὲν κρατέειν ἐθέλει, πάντεσσι δ᾽ ἀνάσσειν,  } \\
{\large\g πᾶσι δὲ σημαίνειν, ἅ τιν᾽ οὐ πείσεσθαι ὀΐω.  } \\
{\large\g εἰ δέ μιν αἰχμητὴν ἔθεσαν θεοὶ αἰὲν ἐόντες,  } \\
{\large\g τοὔνεκά οἱ προθέουσιν ὀνείδεα μυθήσασθαι;  } \\
{\large\g τὸν δ᾽ ἄρ᾽ ὑποβλήδην ἠμείβετο δῖος Ἀχιλλεύς:  } \\
{\large\g ἦ γάρ κεν δειλός τε καὶ οὐτιδανὸς καλεοίμην,  } \\
{\large\g εἰ δὴ σοὶ πᾶν ἔργον ὑπείξομαι ὅττί κεν εἴπηις.  } \\
{\large\g ἄλλοισιν δὴ ταῦτ᾽ ἐπιτέλλεο, μὴ γὰρ ἐμοί γε  } \\
{\large\g σήμαιν᾽: οὐ γὰρ έγώ γ᾽ ἔτι σοὶ πείσεσθαι ὀΐω.  } \\
{\large\g ἄλλο δέ τοι ἐρέω, σὺ δ᾽ ἐνὶ φρεσὶ βάλλεο σῆισιν:  } \\
{\large\g χερσὶ μὲν οὔ τοι ἐγώ γε μαχήσομαι εἵνεκα κούρης,  } \\
{\large\g οὔτε σοὶ οὔτε τωι ἄλλωι, ἐπεί μ᾽ ἀφέλεσθέ γε δόντες:  } \\
{\large\g τῶν δ᾽ ἄλλων ἅ μοί ἐστι θοῆι παρὰ νηῒ μελαίνηι,  } \\
{\large\g τῶν οὐκ ἄν τι φέροις ἀνελὼν ἀέκοντος ἐμεῖο.  } \\
{\large\g εἰ δ᾽ ἄγε μὴν πείρησαι, ἵνα γνώωσι καὶ οἵδε:  } \\
{\large\g αἶψά τοι αἷμα κελαινὸν ἐρωήσει περὶ δουρί.  } \\
{\large\g ὣς τώ γ᾽ ἀντιβίοισι μαχεσσαμένω ἐπέεσσιν  } \\
{\large\g ἀνστήτην, λῦσαν δ᾽ ἀγορὴν παρὰ νηυσὶν Ἀχαιῶν.  } \\
{\large\g Πηλεΐδης μὲν ἐπὶ κλισίας καὶ νῆας ἐΐσας  } \\
{\large\g ἤϊε σύν τε Μενοιτιάδηι καὶ οἷς ἑτάροισιν:  } \\
{\large\g Ἀτρεΐδης δ᾽ ἄρα νῆα θοὴν ἅλαδε προέρυσσεν,  } \\
{\large\g ἐν δ᾽ ἐρέτας ἔκρινεν ἐείκοσιν, ἐς δ᾽ ἑκατόμβην  } \\
{\large\g βῆσε θεῶι, ἀνὰ δὲ Χρυσηΐδα καλλιπάρηον  } \\
{\large\g εἷσεν ἄγων: ἐν δ᾽ ἀρχὸς ἔβη πολύμητις Ὀδυσσεύς.  } \\
{\large\g οἳ μὲν ἔπειτ᾽ ἀναβάντες ἐπέπλεον ὑγρὰ κέλευθα,  } \\
{\large\g λαοὺς δ᾽ Ἀτρεΐδης ἀπολυμαίνεσθαι ἄνωγεν.  } \\
{\large\g οἳ δ᾽ ἀπελυμαίνοντο καὶ εἰς ἅλα λύματ᾽ ἔβαλλον,  } \\
{\large\g ἕρδον δ᾽ Ἀπόλλωνι τεληέσσας ἑκατόμβας  } \\
{\large\g ταύρων ἠδ᾽ αἰγῶν παρὰ θῖν᾽ ἁλὸς ἀτρυγέτοιο:  } \\
{\large\g κνίση δ᾽ οὐρανὸν ἷκεν ἑλισσομένη περὶ καπνῶι.  } \\
{\large\g ὣς οἳ μὲν τὰ πένοντο κατὰ στρατόν: οὐδ᾽ Ἀγαμέμνων  } \\
{\large\g λῆγ᾽ ἔριδος, τὴν πρῶτον ἐπηπείλησ᾽ Ἀχιλῆϊ,  } \\
{\large\g ἀλλ᾽ ὅ γε Ταλθύβιόν τε καὶ Εὐρυβάτην προσέειπεν,  } \\
{\large\g τώ οἱ ἔσαν κήρυκε καὶ ὀτρηρὼ θεράποντε:  } \\
{\large\g ἔρχεσθον κλισίην Πηληϊάδεω Ἀχιλῆος.  } \\
{\large\g χειρὸς ἑλόντ᾽ ἀγέμεν Βρισηΐδα καλλιπάρηον.  } \\
{\large\g εἰ δέ κε μὴ δώησιν, ἐγὼ δέ κεν αὐτὸς ἕλωμαι  } \\
{\large\g ἐλθὼν σὺν πλεόνεσσι, τό οἱ καὶ ῥίγιον ἔσται.  } \\
{\large\g ὣς εἰπὼν προΐει, κρατερὸν δ᾽ ἐπὶ μῦθον ἔτελλεν.  } \\
{\large\g τὼ δ᾽ ἀέκοντε βάτην παρὰ θῖν᾽ ἁλὸς ἀτρυγέτοιο,  } \\
{\large\g Μυρμιδόνων δ᾽ ἐπί τε κλισίας καὶ νῆας ἱκέσθην.  } \\
{\large\g τὸν δ᾽ εὗρον παρά τε κλισίηι καὶ νηῒ μελαίνηι  } \\
{\large\g ἥμενον: οὐδ᾽ ἄρα τώ γε ἰδὼν γήθησεν Ἀχιλλεύς.  } \\
{\large\g τὼ μὲν ταρβήσαντε καὶ αἰδομένω βασιλῆα  } \\
{\large\g στήτην, οὐδέ τί μιν προσεφώνεον οὐδ᾽ ἐρέοντο:  } \\
{\large\g αὐτὰρ ὃ ἔγνω  ἧισιν ἐνὶ φρεσὶ φώνησέν τε:  } \\
{\large\g χαίρετε κήρυκες, Διὸς ἄγγελοι ἠδὲ καὶ ἀνδρῶν,  } \\
{\large\g ἆσσον ἴτ᾽: οὔ τί μοι ὔμμες ἐπαίτιοι, ἀλλ᾽ Ἀγαμέμνων,  } \\
{\large\g ὃ σφῶϊ προΐει Βρισηΐδος εἵνεκα κούρης.  } \\
{\large\g ἀλλ᾽ ἄγε, διογενὲς Πατρόκλεις, ἔξαγε κούρην  } \\
{\large\g καί σφωϊν δὸς ἄγειν. τὼ δ᾽ αὐτὼ μάρτυροι ἔστων  } \\
{\large\g πρός τε θεῶν μακάρων πρός τε θνητῶν ἀνθρώπων  } \\
{\large\g καὶ πρὸς τοῦ βασιλῆος ἀπηνέος, εἴ ποτε δ᾽ αὖτε  } \\
{\large\g χρειὼ ἐμεῖο γένηται ἀεικέα λοιγὸν ἀμῦναι  } \\
{\large\g τοῖς ἄλλοις. ἦ γὰρ ὅ γ᾽ ὀλοιῆισι φρεσὶ θύει,  } \\
{\large\g οὐδέ τι οἶδε νοῆσαι ἅμα πρόσσω καὶ ὀπίσσω,  } \\
{\large\g ὅππως οἱ παρὰ νηυσὶ σόοι μαχέοιατ᾽ Ἀχαιοί.  } \\
{\large\g ὣς φάτο, Πάτροκλος δὲ φίλωι ἐπεπείθεθ᾽ ἑταίρωι,  } \\
{\large\g ἐκ δ᾽ ἄγαγε κλισίης Βρισηΐδα καλλιπάρηον,  } \\
{\large\g δῶκε δ᾽ ἄγειν. τὼ δ᾽ αὖτις ἴτην παρὰ νῆας Ἀχαιῶν,  } \\
{\large\g ἣ δ᾽ ἀέκουσ᾽ ἅμα τοῖσι γυνὴ κίεν. αὐτὰρ Ἀχιλλεὺς  } \\
{\large\g δακρύσας ἑτάρων ἄφαρ ἕζετο νόσφι λιασθείς  } \\
{\large\g θῖν᾽ ἔφ᾽ ἁλὸς πολιῆς, ὁρόων ἐπὶ οἴνοπα πόντον:  } \\
{\large\g πολλὰ δὲ μητρὶ φίληι ἠρήσατο χεῖρας ὀρεγνύς:  } \\
{\large\g μῆτερ, ἐπεί μ᾽ ἔτεκές γε μινυνθάδιόν περ ἐόντα,  } \\
{\large\g τιμήν πέρ μοι ὄφελλεν Ὀλύμπιος ἐγγυαλίξαι  } \\
{\large\g Ζεὺς ὑψιβρεμέτης: νῦν δ᾽ οὐδέ με τυτθὸν ἔτισεν.  } \\
{\large\g ἦ γάρ μ᾽ Ἀτρεΐδης εὐρὺ κρείων Ἀγαμέμνων  } \\
{\large\g ἠτίμησεν, ἑλὼν γὰρ ἔχει γέρας αὐτὸς ἀπούρας.  } \\
{\large\g ὣς φάτο δάκρυ χέων, τοῦ δ᾽ ἔκλυε πότνια μήτηρ  } \\
{\large\g ἡμένη ἐν βένθεσσιν ἁλὸς παρὰ πατρὶ γέροντι.  } \\
{\large\g καρπαλίμως δ᾽ ἀνέδυ πολιῆς ἁλὸς ἠΰτ᾽ ὀμίχλη,  } \\
{\large\g καί ῥα πάροιθ᾽ αὐτοῖο καθέζετο δάκρυ χέοντος,  } \\
{\large\g χειρί τέ μιν κατέρεξεν, ἔπος τ᾽ ἔφατ᾽ ἔκ τ᾽ ὀνόμαζεν:  } \\
{\large\g τέκνον, τί κλαίεις; τί δέ σε φρένας ἵκετο πένθος;  } \\
{\large\g ἐξαύδα, μὴ κεῦθε νόωι, ἵνα εἴδομεν ἄμφω.  } \\
{\large\g τὴν δὲ βαρὺ στενάχων προσέφη πόδας ὠκὺς Ἀχιλλεύς:  } \\
{\large\g οἶσθα: τίη τοι ταῦτα ἰδυίηι πάντ᾽ ἀγορεύω;  } \\
{\large\g ωἰχόμεθ᾽ ἐς Θήβην, ἱερὴν πόλιν Ἠετίωνος,  } \\
{\large\g τὴν δὲ διεπράθομέν τε καὶ ἤγομεν ἐνθάδε πάντα.  } \\
{\large\g καὶ τὰ μὲν εὖ δάσσαντο μετὰ σφίσιν υἷες Ἀχαιῶν,  } \\
{\large\g ἐκ δ᾽ ἕλον Ἀτρεΐδηι Χρυσηΐδα καλλιπάρηον.  } \\
{\large\g Χρύσης δ᾽ αὖθ᾽, ἱερεὺς ἑκατηβόλου Ἀπόλλωνος,  } \\
{\large\g ἦλθε θοὰς ἐπὶ νῆας Ἀχαιῶν χαλκοχιτώνων  } \\
{\large\g λυσόμενός τε θύγατρα φέρων τ᾽ ἀπερείσι᾽ ἄποινα,  } \\
{\large\g στέμματ᾽ ἔχων ἐν χερσὶν ἑκηβόλου Ἀπόλλωνος  } \\
{\large\g χρυσέωι ἀνὰ σκήπτρωι, καὶ λίσσετο πάντας Ἀχαιούς,  } \\
{\large\g Ἀτρεΐδα δὲ μάλιστα δύω, κοσμήτορε λαῶν.  } \\
{\large\g  ἔνθ᾽ ἄλλοι μὲν πάντες ἐπευφήμησαν Ἀχαιοί,  } \\
{\large\g αἰδεῖσθαί θ᾽ ἱερῆα καὶ ἀγλαὰ δέχθαι ἄποινα:  } \\
{\large\g  ἀλλ᾽ οὐκ Ἀτρεΐδηι Ἀγαμέμνονι ἥνδανε θυμῶι,  } \\
{\large\g  ἀλλὰ κακῶς ἀφίει, κρατερὸν δ᾽ ἐπὶ μῦθον ἔτελλεν.  } \\
{\large\g χωόμενος δ᾽ ὁ γέρων πάλιν ὤιχετο: τοῖο δ᾽ Ἀπόλλων  } \\
{\large\g εὐξαμένου ἤκουσεν, ἐπεὶ μάλα οἱ φίλος ἦεν,  } \\
{\large\g ἧκε δ᾽ ἐπ᾽ Ἀργείοισι κακὸν βέλος: οἳ δέ νυ λαοί   } \\
{\large\g θνῆισκον ἐπασσύτεροι, τὰ δ᾽ ἐπώιχετο κῆλα θεοῖο  } \\
{\large\g πάντηι ἀνὰ στρατὸν εὐρὺν Ἀχαιῶν: ἄμμι δὲ μάντις  } \\
{\large\g εὖ εἰδὼς ἀγόρευε θεοπροπίας Ἑκάτοιο.  } \\
{\large\g αὐτίκ᾽ ἐγὼ πρῶτος κελόμην θεὸν ἱλάσκεσθαι:  } \\
{\large\g Ἀτρεΐωνα δ᾽ ἔπειτα χόλος λάβεν, αἶψα δ᾽ ἀναστὰς  } \\
{\large\g ἠπείλησεν μῦθον, ὃ δὴ τετελεσμένος ἐστίν.  } \\
{\large\g τὴν μὲν γὰρ σὺν νηῒ θοῆι ἑλίκωπες Ἀχαιοί  } \\
{\large\g ἐς Χρύσην πέμπουσιν, ἄγουσι δὲ δῶρα ἄνακτι:  } \\
{\large\g τὴν δὲ νέον κλισίηθεν ἔβαν κήρυκες ἄγοντες  } \\
{\large\g κούρην Βρισῆος, τήν μοι δόσαν υἷες Ἀχαιῶν.  } \\
{\large\g ἀλλὰ σύ, εἰ δύνασαί γε, περίσχεο παιδὸς ἑῆος:  } \\
{\large\g ἐλθοῦσ᾽ Οὔλυμπόνδε Δία λίσαι, εἴ ποτε δή τι  } \\
{\large\g ἢ ἔπει ὤνησας κραδίην Διὸς ἠὲ καὶ ἔργωι.  } \\
{\large\g πολλάκι γάρ σεο πατρὸς ἐνὶ μεγάροισιν ἄκουσα  } \\
{\large\g εὐχομένης, ὅτ᾽ ἔφησθα κελαινεφέϊ Κρονίωνι  } \\
{\large\g οἴη ἐν ἀθανάτοισιν ἀεικέα λοιγὸν ἀμῦναι,  } \\
{\large\g ὁππότε μιν ξυνδῆσαι Ὀλύμπιοι ἤθελον ἄλλοι,  } \\
{\large\g Ἥρη τ᾽ ἠδὲ Ποσειδάων καὶ Παλλὰς Ἀθήνη,  } \\
{\large\g ἀλλὰ σὺ τόν γ᾽ ἐλθοῦσα, θεά, ὑπελύσαο δεσμῶν,  } \\
{\large\g ὦχ᾽ ἑκατόγχειρον καλέσασ᾽ ἐς μακρὸν Ὄλυμπον,  } \\
{\large\g ὃν Βριάρεων καλέουσι θεοί, ἄνδρες δέ τε πάντες  } \\
{\large\g Αἰγαίων᾽, ὃ γὰρ αὖτε βίην οὗ πατρὸς ἀμείνων:  } \\
{\large\g ὅς ῥα παρὰ Κρονίωνι καθέζετο κύδεϊ γαίων:  } \\
{\large\g τὸν καὶ ὑπέδδεισαν μάκαρες θεοί, οὐδ᾽ ἔτ᾽ ἔδησαν.  } \\
{\large\g τῶν νῦν μιν μνήσασα παρέζεο καὶ λαβὲ γούνων,  } \\
{\large\g αἴ κέν πως ἐθέληισιν ἐπὶ Τρώεσσιν ἀρῆξαι,  } \\
{\large\g τοὺς δὲ κατὰ πρύμνας τε καὶ ἀμφ᾽ ἅλα ἔλσαι Ἀχαιοὺς  } \\
{\large\g κτεινομένους, ἵνα πάντες ἐπαύρωνται βασιλῆος,  } \\
{\large\g γνῶι δὲ καὶ Ἀτρεΐδης εὐρὺ κρείων Ἀγαμέμνων  } \\
{\large\g ἣν ἄτην, ὅ τ᾽ ἄριστον Ἀχαιῶν οὐδὲν ἔτισεν.  } \\
{\large\g τὸν δ᾽ ἠμείβετ᾽ ἔπειτα Θέτις κατὰ δάκρυ χέουσα:  } \\
{\large\g ὤι μοι τέκνον ἐμόν, τί νύ σ᾽ ἔτρεφον αἰνὰ τεκοῦσα;  } \\
{\large\g αἴθ᾽ ὄφελες παρὰ νηυσὶν ἀδάκρυτος καὶ ἀπήμων  } \\
{\large\g ἧσθαι, ἐπεί νύ τοι αἶσα μίνυνθά περ, οὔ τι μάλα δήν.  } \\
{\large\g νῦν δ᾽ ἅμα τ᾽ ὠκύμορος καὶ ὀϊζυρὸς περὶ πάντων  } \\
{\large\g ἔπλεο. τώ σε κακῆι αἴσηι τέκον ἐν μεγάροισι.  } \\
{\large\g τοῦτο δέ τοι ἐρέουσα ἔπος Διῒ τερπικεραύνωι  } \\
{\large\g εἶμ᾽ αὐτὴ πρὸς Ὄλυμπον ἀγάννιφον, αἴ κε πίθηται.  } \\
{\large\g ἀλλὰ σὺ μὲν νῦν νηυσὶ παρήμενος ὠκυπόροισι  } \\
{\large\g μήνι᾽ Ἀχαιοῖσιν, πολέμου δ᾽ ἀποπαύεο πάμπαν.  } \\
{\large\g Ζεὺς γὰρ ἐς Ὠκεανὸν μετ᾽ ἀμύμονας Αἰθιοπῆας  } \\
{\large\g χθιζὸς ἔβη κατὰ δαῖτα, θεοὶ δ᾽ ἅμα πάντες ἕποντο:  } \\
{\large\g δωδεκάτηι δέ τοι αὖτις ἐλεύσεται Οὔλυμπόνδε:  } \\
{\large\g καὶ τότ᾽ ἔπειτά τοι εἶμι Διὸς ποτὶ χαλκοβατὲς δῶ  } \\
{\large\g καί μιν γουνάσομαι, καί μιν πείσεσθαι ὀΐω.  } \\
{\large\g ὣς ἄρα φωνήσασ᾽ ἀπεβήσετο, τὸν δ᾽ ἔλιπ᾽ αὐτοῦ  } \\
{\large\g χωόμενον κατὰ θυμὸν ἐϋζώνοιο γυναικός,  } \\
{\large\g τήν ῥα βίηι ἀέκοντος ἀπηύρων. αὐτὰρ Ὀδυσσεύς  } \\
{\large\g ἐς Χρύσην ἵκανεν ἄγων ἱερὴν ἑκατόμβην.  } \\
{\large\g οἳ δ᾽ ὅτε δὴ λιμένος πολυβενθέος ἐντὸς ἵκοντο,  } \\
{\large\g ἱστία μὲν στείλαντο, θέσαν δ᾽ ἐν νηῒ μελαίνηι,  } \\
{\large\g ἱστὸν δ᾽ ἱστοδόκηι πέλασαν προτόνοισιν ὑφέντες  } \\
{\large\g καρπαλίμως, τὴν δ᾽ εἰς ὅρμον προέρεσσαν ἐρετμοῖς.  } \\
{\large\g ἐκ δ᾽ εὐνὰς ἔβαλον, κατὰ δὲ πρυμνήσι᾽ ἔδησαν,  } \\
{\large\g ἐκ δὲ καὶ αὐτοὶ βαῖνον ἐπὶ ῥηγμῖνι θαλάσσης,  } \\
{\large\g ἐκ δ᾽ ἑκατόμβην βῆσαν ἑκηβόλωι Ἀπόλλωνι:  } \\
{\large\g ἐκ δὲ Χρυσηῒς νηὸς βῆ ποντοπόροιο.  } \\
{\large\g τὴν μὲν ἔπειτ᾽ ἐπὶ βωμὸν ἄγων πολύμητις Ὀδυσσεὺς  } \\
{\large\g πατρὶ φίλωι ἐν χερσὶ τίθει καί μιν προσέειπεν:  } \\
{\large\g ὦ Χρύση, πρό μ᾽ ἔπεμψεν ἄναξ ἀνδρῶν Ἀγαμέμνων  } \\
{\large\g παῖδά τε σοὶ ἀγέμεν, Φοίβωι θ᾽ ἱερὴν ἑκατόμβην  } \\
{\large\g ῥέξαι ὑπὲρ Δαναῶν, ὄφρ᾽ ἱλασόμεσθα ἄνακτα,  } \\
{\large\g ὃς νῦν Ἀργείοισι πολύστονα κήδε᾽ ἐφῆκεν.  } \\
{\large\g ὣς εἰπὼν ἐν χερσὶ τίθει, ὃ δὲ δέξατο χαίρων  } \\
{\large\g παῖδα φίλην: τοὶ δ᾽ ὦκα θεῶι κλειτὴν ἑκατόμβην  } \\
{\large\g ἑξείης ἔστησαν ἐΰδμητον περὶ βωμόν,  } \\
{\large\g χερνίψαντο δ᾽ ἔπειτα καὶ οὐλοχύτας ἀνέλοντο.  } \\
{\large\g τοῖσιν δὲ Χρύσης μεγάλ᾽ ηὔχετο χεῖρας ἀνασχών:  } \\
{\large\g  κλῦθί μοι, ἀργυρότοξ᾽, ὃς Χρύσην ἀμφιβέβηκας  } \\
{\large\g  Κίλλάν τε ζαθέην, Τενέδοιό τε ἶφι ἀνάσσεις.  } \\
{\large\g ἠμὲν δή ποτ᾽ ἐμέο πάρος ἔκλυες εὐξαμένοιο:  } \\
{\large\g τίμησας μὲν ἐμέ, μέγα δ᾽ ἴψαο λαὸν Ἀχαιῶν:  } \\
{\large\g ἠδ᾽ ἔτι καὶ νῦν μοι τόδ᾽ ἐπικρήηνον ἐέλδωρ:  } \\
{\large\g ἤδη νῦν Δαναοῖσιν ἀεικέα λοιγὸν ἄμυνον.  } \\
{\large\g ὣς ἔφατ᾽ εὐχόμενος, τοῦ δ᾽ ἔκλυε Φοῖβος Ἀπόλλων.  } \\
{\large\g αὐτὰρ ἐπεί ῥ᾽ εὔξαντο καὶ οὐλοχύτας προβάλοντο,  } \\
{\large\g αὐέρυσαν μὲν πρῶτα καὶ ἔσφαξαν καὶ ἔδειραν,  } \\
{\large\g μηρούς τ᾽ ἐξέταμον κατά τε κνίσηι ἐκάλυψαν  } \\
{\large\g δίπτυχα ποιήσαντες, ἐπ᾽ αὐτῶν δ᾽ ὠμοθέτησαν.  } \\
{\large\g καῖε δ᾽ ἐπὶ σχίζηις ὁ γέρων, ἐπὶ δ᾽ αἴθοπα οἶνον  } \\
{\large\g λεῖβε: νέοι δὲ παρ᾽ αὐτὸν ἔχον πεμπώβολα χερσίν.  } \\
{\large\g αὐτὰρ ἐπεὶ κατὰ μῆρ᾽ ἐκάη καὶ σπλάγχν᾽ ἐπάσαντο,  } \\
{\large\g μίστυλλόν τ᾽ ἄρα τἄλλα καὶ ἀμφ᾽ ὀβελοῖσιν ἔπειραν  } \\
{\large\g ὤπτησάν τε περιφραδέως, ἐρύσαντό τε πάντα.  } \\
{\large\g αὐτὰρ ἐπεὶ παύσαντο πόνου τετύκοντό τε δαῖτα,  } \\
{\large\g δαίνυντ᾽, οὐδέ τι θυμὸς ἐδεύετο δαιτὸς ἐΐσης.  } \\
{\large\g αὐτὰρ ἐπεὶ πόσιος καὶ ἐδητύος ἐξ ἔρον ἕντο,  } \\
{\large\g κοῦροι μὲν κρητῆρας ἐπεστέψαντο ποτοῖο,  } \\
{\large\g νώμησαν δ᾽ ἄρα πᾶσιν ἐπαρξάμενοι δεπάεσσιν,  } \\
{\large\g οἳ δὲ πανημέριοι μολπῆι θεὸν ἱλάσκοντο  } \\
{\large\g καλὸν ἀείδοντες παιήονα κοῦροι Ἀχαιῶν,  } \\
{\large\g μέλποντες ἑκάεργον: ὃ δὲ φρένα τέρπετ᾽ ἀκούων.  } \\
{\large\g ἦμος δ᾽ ἠέλιος κατέδυ καὶ ἐπὶ κνέφας ἦλθεν,  } \\
{\large\g δὴ τότε κοιμήσαντο παρὰ πρυμνήσια νηός:  } \\
{\large\g ἦμος δ᾽ ἠριγένεια φάνη ῥοδοδάκτυλος Ἠώς,  } \\
{\large\g καὶ τότ᾽ ἔπειτ᾽ ἀνάγοντο μετὰ στρατὸν εὐρὺν Ἀχαιῶν.  } \\
{\large\g τοῖσιν δ᾽ ἴκμενον οὖρον ἵει ἑκάεργος Ἀπόλλων,  } \\
{\large\g οἳ δ᾽ ἱστὸν στήσαντ᾽ ἀνά θ᾽ ἱστία λευκὰ πέτασσαν,  } \\
{\large\g ἐν δ᾽ ἄνεμος πρῆσεν μέσον ἱστίον, ἀμφὶ δὲ κῦμα  } \\
{\large\g στείρηι πορφύρεον μεγάλ᾽ ἴαχε νηὸς ἰούσης,  } \\
{\large\g ἣ δ᾽ ἔθεεν κατὰ κῦμα διαπρήσσουσα κέλευθον.  } \\
{\large\g αὐτὰρ ἐπεί ῥ᾽ ἵκοντο μετὰ στρατὸν εὐρὺν Ἀχαιῶν,  } \\
{\large\g νῆα μὲν οἵ γε μέλαιναν ἐπ᾽ ἠπείροιο ἔρυσσαν  } \\
{\large\g ὑψοῦ ἐπὶ ψαμάθοις, ὑπὸ δ᾽ ἕρματα μακρὰ τάνυσσανn  } \\
{\large\g αὐτοὶ δ᾽ ἐσκίδναντο κατὰ κλισίας τε νέας τε.  } \\
{\large\g αὐτὰρ ὃ μήνιε νηυσὶ παρήμενος ὠκυπόροισι  } \\
{\large\g διογενὴς Πηλέως υἱός, πόδας ὠκὺς Ἀχιλλεύς:  } \\
{\large\g οὔτέ ποτ᾽ εἰς ἀγορὴν πωλέσκετο κυδιάνειραν  } \\
{\large\g οὔτέ ποτ᾽ ἐς πόλεμον, ἀλλὰ φθινύθεσκε φίλον κῆρ  } \\
{\large\g αὖθι μένων: ποθέεσκε δ᾽ ἀϋτήν τε πτόλεμόν τε.  } \\
{\large\g ἀλλ᾽ ὅτε δή ῥ᾽ ἐκ τοῖο δυωδεκάτη γένετ᾽ ἠώς,  } \\
{\large\g καὶ τότε δὴ πρὸς Ὄλυμπον ἴσαν θεοὶ αἰὲν ἐόντες  } \\
{\large\g πάντες ἅμα, Ζεὺς δ᾽ ἦρχε: Θέτις δ᾽ οὐ λήθετ᾽ ἐφετμέων  } \\
{\large\g παιδὸς ἑοῦ, ἀλλ᾽ ἥ γ᾽ ἀνεδύσατο κῦμα θαλάσσης,  } \\
{\large\g ἠερίη δ᾽ ἀνέβη μέγαν οὐρανὸν Οὔλυμπόν τε.  } \\
{\large\g ηὗρεν δ᾽ εὐρύοπα Κρονίδην ἄτερ ἥμενον ἄλλων  } \\
{\large\g ἀκροτάτηι κορυφῆι πολυδειράδος Οὐλύμποιο:  } \\
{\large\g καί ῥα πάροιθ᾽ αὐτοῖο καθέζετο, καὶ λάβε γούνων  } \\
{\large\g σκαιῆι, δεξιτερῆι δ᾽ ἄρ᾽ ὑπ᾽ ἀνθερεῶνος ἑλοῦσα  } \\
{\large\g λισσομένη προσέειπε Δία Κρονίωνα ἄνακτα:  } \\
{\large\g Ζεῦ πάτερ εἴ ποτε δή σε μετ᾽ ἀθανάτοισιν ὄνησα  } \\
{\large\g ἢ ἔπει ἢ ἔργωι, τόδε μοι κρήηνον ἐέλδωρ:  } \\
{\large\g τίμησόν μοι υἱόν, ὃς ὠκυμορώτατος ἄλλων  } \\
{\large\g ἔπλετ᾽: ἀτάρ μιν νῦν γε ἄναξ ἀνδρῶν Ἀγαμέμνων  } \\
{\large\g ἠτίμησεν: ἑλὼν γὰρ ἔχει γέρας αὐτὸς ἀπούρας.  } \\
{\large\g ἀλλὰ σύ πέρ μιν τῖσον, Ὀλύμπιε μητίετα Ζεῦ:  } \\
{\large\g τόφρα δ᾽ ἐπὶ Τρώεσσι τίθει κράτος, ὄφρ᾽ ἂν Ἀχαιοί  } \\
{\large\g υἱὸν ἐμὸν τίσωσιν ὀφέλλωσίν τέ ἑ τιμῆι.  } \\
{\large\g ὣς φάτο: τὴν δ᾽ οὔ τι προσέφη νεφεληγερέτα Ζεύς,  } \\
{\large\g ἀλλ᾽ ἀκέων δὴν ἧστο. Θέτις δ᾽ ὡς ἥψατο γούνων,  } \\
{\large\g ὣς ἔχετ᾽ ἐμπεφυυῖα, καὶ εἴρετο δεύτερον αὖτις:  } \\
{\large\g νημερτὲς μὲν δή μοι ὑπόσχεο καὶ κατάνευσον,  } \\
{\large\g ἠ᾽ ἀπόειπ᾽, ἐπεὶ οὔ τοι ἔπι δέος, ὄφρ᾽ εὖ εἴδω  } \\
{\large\g ὅσσον ἐγὼ μετὰ πᾶσιν ἀτιμοτάτη θεός εἰμι.  } \\
{\large\g τὴν δὲ μέγ᾽ ὀχθήσας προσέφη νεφεληγερέτα Ζεύς:  } \\
{\large\g ἦ δὴ λοίγια ἔργ᾽, ὅ τέ μ᾽ ἐχθοδοπῆσαι ἐφήσεις  } \\
{\large\g Ἥρηι, ὅτ᾽ ἄν μ᾽ ἐρέθηισιν ὀνειδείοις ἐπέεσσιν.  } \\
{\large\g ἣ δὲ καὶ αὔτως μ᾽ αἰεὶ ἐν ἀθανάτοισι θεοῖσι  } \\
{\large\g νεικεῖ, καί τέ μέ φησι μάχηι Τρώεσσιν ἀρήγειν.  } \\
{\large\g ἀλλὰ σὺ μὲν νῦν αὖτις ἀπόστιχε, μή τι νοήσηι  } \\
{\large\g Ἥρη: ἐμοὶ δέ κε ταῦτα μελήσεται, ὄφρα τελέσσω.  } \\
{\large\g εἰ δ᾽ ἄγε τοι κεφαλῆι κατανεύσομαι, ὄφρα πεποίθηις:  } \\
{\large\g τοῦτο γὰρ ἐξ ἐμέθεν γε μετ᾽ ἀθανάτοισι μέγιστον  } \\
{\large\g τέκμωρ: οὐ γὰρ ἐμὸν παλινάγρετον οὐδ᾽ ἀπατηλόν   } \\
{\large\g οὐδ᾽ ἀτελεύτητον, ὅ τί κεν κεφαλῆι κατανεύσω.  } \\
{\large\g ἦ καὶ κυανέηισιν ἐπ᾽ ὀφρύσι νεῦσε Κρονίων,  } \\
{\large\g ἀμβρόσιαι δ᾽ ἄρα χαῖται ἐπερρώσαντο ἄνακτος  } \\
{\large\g κρατὸς ἀπ᾽ ἀθανάτοιο: μέγαν δ᾽ ἐλέλιξεν Ὄλυμπον.  } \\
{\large\g τώ γ᾽ ὣς βουλεύσαντε διέτμαγεν: ἣ μὲν ἔπειτα  } \\
{\large\g εἰς ἅλα ἄλτο βαθεῖαν ἀπ᾽ αἰγλήεντος Ὀλύμπου,  } \\
{\large\g Ζεὺς δὲ ἑὸν πρὸς δῶμα: θεοὶ δ᾽ ἅμα πάντες ἀνέσταν  } \\
{\large\g ἐξ ἑδέων σφοῦ πατρὸς ἐναντίον, οὐδέ τις ἔτλη  } \\
{\large\g μεῖναι ἐπερχόμενον, ἀλλ᾽ ἀντίοι ἔσταν ἅπαντες.  } \\
{\large\g ὣς ὃ μὲν ἔνθα καθέζετ᾽ ἐπὶ θρόνου: οὐδέ μιν Ἥρη  } \\
{\large\g ἠγνοίησεν ἰδοῦσ᾽ ὅτι οἱ συμφράσσατο βουλάς   } \\
{\large\g ἀργυρόπεζα Θέτις, θυγάτηρ ἁλίοιο γέροντος.  } \\
{\large\g αὐτίκα κερτομίοισι Δία Κρονίωνα προσηύδα:  } \\
{\large\g τίς δὴ αὖ τοι, δολομῆτα, θεῶν συμφράσσατο βουλάς;  } \\
{\large\g αἰεί τοι φίλον ἐστὶν ἐμεῖ᾽ ἀπὸ νόσφιν ἐόντα  } \\
{\large\g κρυπτάδια φρονέοντα δικαζέμεν: οὐδέ τί πώ μοι  } \\
{\large\g πρόφρων τέτληκας εἰπεῖν ἔπος ὅττι νοήσηις.  } \\
{\large\g τὴν δ᾽ ἠμείβετ᾽ ἔπειτα πατὴρ ἀνδρῶν τε θεῶν τε:  } \\
{\large\g Ἥρη, μὴ δὴ πάντας ἐμοὺς ἐπιέλπεο μύθους  } \\
{\large\g εἰδήσειν: χαλεποί τοι ἔσοντ᾽ ἀλόχωι περ ἐούσηι:  } \\
{\large\g ἀλλ᾽ ὃν μέν κ᾽ ἐπιεικὲς ἀκουέμεν, οὔ τις ἔπειτα  } \\
{\large\g οὔτε θεῶν πρότερος τόν γ᾽ εἴσεται οὔτ᾽ ἀνθρώπων:  } \\
{\large\g ὃν δέ κ᾽ ἐγὼν ἀπάνευθε θεῶν ἐθέλωμι νοῆσαι,  } \\
{\large\g μή τι σὺ ταῦτα ἕκαστα διείρεο μηδὲ μετάλλα.  } \\
{\large\g τὸν δ᾽ ἠμείβετ᾽ ἔπειτα βοῶπις πότνια Ἥρη:  } \\
{\large\g αἰνότατε Κρονίδη, ποῖον τὸν μῦθον ἔειπες;  } \\
{\large\g καὶ λίην σε πάρος γ᾽ οὔτ᾽ εἴρομαι οὔτε μεταλλῶ,  } \\
{\large\g ἀλλὰ μάλ᾽ εὔκηλος τὰ φράζεαι ἅσσ᾽ ἐθέληισθα.  } \\
{\large\g νῦν δ᾽ αἰνῶς δείδοικα κατὰ φρένα, μή σε παρείπηι  } \\
{\large\g ἀργυρόπεζα Θέτις, θυγάτηρ ἁλίοιο γέροντος:  } \\
{\large\g ἠερίη γὰρ σοί γε παρέζετο καὶ λάβε γούνων.  } \\
{\large\g τῆι σ᾽ ὀΐω κατανεῦσαι ἐτήτυμον, ὡς Ἀχιλῆα  } \\
{\large\g τιμήσηις, ὀλέσηις δὲ πολὺς ἐπὶ νηυσὶν Ἀχαιῶν.  } \\
{\large\g τὴν δ᾽ ἀπαμειβόμενος προσέφη νεφεληγερέτα Ζεύς:  } \\
{\large\g δαιμονίη, αἰεὶ μὲν ὀΐεαι, οὐδέ σε λήθω,  } \\
{\large\g πρῆξαι δ᾽ ἔμπης οὔ τι δυνήσεαι, ἀλλ᾽ ἀπὸ θυμοῦ  } \\
{\large\g μᾶλλον ἐμοὶ ἔσεαι, τὸ δέ τοι καὶ ῥίγιον ἔσται.  } \\
{\large\g εἰ δ᾽ οὕτω τοῦτ᾽ ἐστὶν, ἐμοὶ μέλλει φίλον εἶναι.  } \\
{\large\g ἀλλ᾽ ἀκέουσα κάθησο, ἐμῶι δ᾽ ἐπιπείθεο μύθωι,  } \\
{\large\g μή νύ τοι οὐ χραίσμωσιν ὅσοι θεοί εἰσ᾽ ἐν Ὀλύμπωι  } \\
{\large\g ἆσσον ἰόνθ᾽, ὅτε κέν τοι ἀάπτους χεῖρας ἐφείω.  } \\
{\large\g ὣς ἔφατ᾽ ἔδδεισεν δὲ βοῶπις πότνια Ἥρη,  } \\
{\large\g καί ῥ᾽ ἀκέουσα καθῆστο ἐπιγνάμψασα φίλον κῆρ:  } \\
{\large\g ὄχθησαν δ᾽ ἀνὰ δῶμα Διὸς θεοὶ Οὐρανίωνες.  } \\
{\large\g τοῖσιν δ᾽ Ἥφαιστος κλυτοτέχνης ἦρχ᾽ ἀγορεύειν,  } \\
{\large\g μητρὶ φίληι ἐπίηρα φέρων, λευκωλένωι Ἥρηι:  } \\
{\large\g ἦ δὴ λοίγια ἔργα τάδ᾽ ἔσσεται, οὐδ᾽ ἔτ᾽ ἀνεκτά,  } \\
{\large\g εἰ δὴ σφὼ ἕνεκα θνητῶν ἐριδαίνετον ὧδε,  } \\
{\large\g ἐν δὲ θεοῖσι κολωιὸν ἐλαύνετον: οὐδέ τι δαιτός  } \\
{\large\g ἐσθλῆς ἔσσεται ἦδος, ἐπεὶ τὰ χερείονα νικᾶι.  } \\
{\large\g μητρὶ δ᾽ ἐγὼ παράφημι, καὶ αὐτῆι περ νοεούσηι,  } \\
{\large\g πατρὶ φίλωι ἐπίηρα φέρειν Διΐ, ὄφρα μὴ αὖτε  } \\
{\large\g νεικείηισι πατήρ, σὺν δ᾽ ἡμῖν δαῖτα ταράξηι.  } \\
{\large\g εἴ περ γάρ κ᾽ ἐθέληισιν Ὀλύμπιος ἀστεροπητής  } \\
{\large\g ἐξ ἑδέων στυφελίξαι: ὃ γὰρ πολὺ φέρτατός ἐστιν.  } \\
{\large\g ἀλλὰ σὺ τόν γ᾽ ἐπέεσσι καθάπτεσθαι μαλακοῖσιν:  } \\
{\large\g αὐτίκ᾽ ἔπειθ᾽ ἵλαος Ὀλύμπιος ἔσσεται ἡμῖν.  } \\
{\large\g ὣς ἄρ᾽ ἔφη, καὶ ἀναΐξας δέπας ἀμφικύπελλον  } \\
{\large\g μητρὶ φίληι ἐν χερσὶ τίθει, καί μιν προσέειπε:  } \\
{\large\g τέτλαθι, μῆτερ ἐμή, καὶ ἀνάσχεο κηδομένη περ,  } \\
{\large\g μή σε φίλην περ ἐοῦσαν ἐν ὀφθαλμοῖσιν ἴδωμαι  } \\
{\large\g θεινομένην, τότε δ᾽ οὔ τι δυνήσομαι, ἀχνύμενός περ,  } \\
{\large\g χραισμεῖν: ἀργαλέος γὰρ Ὀλύμπιος ἀντιφέρεσθαι.  } \\
{\large\g ἤδη γάρ με καὶ ἄλλοτ᾽ ἀλεξέμεναι μεμαῶτα  } \\
{\large\g ῥῖψε ποδὸς τεταγὼν ἀπὸ βηλοῦ θεσπεσίοιο:  } \\
{\large\g πᾶν δ᾽ ἦμαρ φερόμην, ἅμα δ᾽ ἠελίωι καταδύντι  } \\
{\large\g κάππεσον ἐν Λήμνωι, ὀλίγος δ᾽ ἔτι θυμὸς ἐνῆεν.  } \\
{\large\g ἔνθά με Σίντιες ἄνδρες ἄφαρ κομίσαντο πεσόντα.  } \\
{\large\g ὣς φάτο, μείδησεν δὲ θεὰ λευκώλενος Ἥρη,  } \\
{\large\g μειδήσασα δὲ παιδὸς ἐδέξατο χειρὶ κύπελλον.  } \\
{\large\g αὐτὰρ ὃ τοῖς ἄλλοισι θεοῖς ἐνδέξια πᾶσιν  } \\
{\large\g οἰνοχόει γλυκὺ νέκταρ ἀπὸ κρητῆρος ἀφύσσων:  } \\
{\large\g ἄσβεστος δ᾽ ἄρ᾽ ἐνῶρτο γέλως μακάρεσσι θεοῖσιν,  } \\
{\large\g ὡς ἴδον Ἥφαιστον διὰ δώματα ποιπνύοντα.  } \\
{\large\g ὣς τότε μὲν πρόπαν ἦμαρ ἐς ἠέλιον καταδύντα  } \\
{\large\g δαίνυντ᾽, οὐδέ τι θυμὸς ἐδεύετο δαιτὸς ἐΐσης,  } \\
{\large\g οὐ μὲν φόρμιγγος περικαλλέος ἣν ἔχ᾽ Ἀπόλλων,  } \\
{\large\g Μουσάων θ᾽ αἳ ἄειδον ἀμειβόμεναι ὀπὶ καλῆι,  } \\
{\large\g αὐτὰρ ἐπεὶ κατέδυ λαμπρὸν φάος ἠελίοιο,  } \\
{\large\g οἳ μὲν κακκείοντες ἔβαν οἶκόνδε ἕκαστος,  } \\
{\large\g ἧχι ἑκάστωι δῶμα περικλυτὸς ἀμφιγυήεις  } \\
{\large\g Ἥφαιστος ποίησεν ἰδυίηισι πραπίδεσσι:  } \\
{\large\g Ζεὺς δὲ πρὸς ὃν λέχος ἤϊ᾽ Ὀλύμπιος ἀστεροπητής,  } \\
{\large\g ἔνθα πάρος κοιμᾶθ᾽ ὅτε μιν γλυκὺς ὕπνος ἱκάνοι:  } \\
{\large\g ἔνθα καθεῦδ᾽ ἀναβάς, παρὰ δὲ χρυσόθρονος Ἥρη.  } \\
\end{verse}  % book one 
\end{Spacing}


